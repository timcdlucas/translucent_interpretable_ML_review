%%%%%%%%%%%%%%%%%%%%%%%%%%%%%%%%%%%%%%%%%
% Professional Formal Letter
% LaTeX Template
% Version 2.0 (12/2/17)
%
% This template originates from:
% http://www.LaTeXTemplates.com
%
% Authors:
% Brian Moses
% Vel (vel@LaTeXTemplates.com)
%
% License:
% CC BY-NC-SA 3.0 (http://creativecommons.org/licenses/by-nc-sa/3.0/)
%
%%%%%%%%%%%%%%%%%%%%%%%%%%%%%%%%%%%%%%%%

%----------------------------------------------------------------------------------------
%	PACKAGES AND OTHER DOCUMENT CONFIGURATIONS
%----------------------------------------------------------------------------------------

\documentclass[10pt, a4paper]{letter} % Set the font size (10pt, 11pt and 12pt) and paper size (letterpaper, a4paper, etc)


\usepackage[nodayofweek]{datetime}


\usepackage[hidelinks]{hyperref}

\setlength\longindentation{5cm}

\input{structure.tex} % Include the file that specifies the document structure

%\longindentation=0pt % Un-commenting this line will push the closing "Sincerely," and date to the left of the page

%----------------------------------------------------------------------------------------
%	YOUR INFORMATION
%----------------------------------------------------------------------------------------

\Who{Dr Tim C.D. Lucas} % Your name

\Title{} % Your title, leave blank for no title

\authordetails{
	Big Data Institute\\ % Your department/institution
	University of Oxford\\
	OX3 7LF,, U.K.\\ % Your city, zip code, country, etc
	timcdlucas@gmail.com\\ % Your email address
	+44 (0) 7415863536\\ % Your phone number
	\\
	\today
}

%----------------------------------------------------------------------------------------
%	HEADER CONTENTS
%----------------------------------------------------------------------------------------

\logo{oxford-logo.png} % Logo filename, your logo should have square dimensions (i.e. roughly the same width and height), if it does not, you will need to adjust spacing within the HEADER STRUCTURE block in structure.tex (read the comments carefully!)

\headerlineone{UNIVERSITY} % Top header line, leave blank if you only want the bottom line

\headerlinetwo{OF OXFORD} % Bottom header line

%----------------------------------------------------------------------------------------

\begin{document}

%----------------------------------------------------------------------------------------
%	TO ADDRESS
%----------------------------------------------------------------------------------------

\begin{letter}{
Prof. Aimeé T. Classen
Editor-in-Chief
Ecological Monographs
}

%----------------------------------------------------------------------------------------
%	LETTER CONTENT
%----------------------------------------------------------------------------------------

\opening{Dear Prof. Aimeé T. Classen,}

I am pleased to submit the article ``A translucent box: interpretable machine learning in ecology'' to be considered for publication as a ``Review'' paper in Ecological Monographs.
There are two files: the main manuscript (lucas-translucent-ms.pdf) and the the supplementary code (lucas-translucent-code.R).

Machine learning has established itself as an important tool in ecology.
However, its use has been restricted to predictive tasks and the models are seen as a black box.
There is, however, scope for machine learning models to be interpreted.
The interpretation of these models can be used to increase trust in the models for both the analyst and other stakeholders that are using predictions from the models for example in applied activities such as conservation or restoration.
Alternatively, interpreting the models can be an end in itself, as a method for exploratory data analysis for example.

In this review I cover the state of the field of interpretable machine learning.
I describe methods that can be used to interpret the model at the level of the system, the variable or the individual prediction.
In particular I describe methods for incorporating and interpreting random effects (categorical, spatial, phylogenetic and temporal) in machine learning models.
I demonstrate all of these methods in a case study of predicting litter size in a large database of mammal traits though the methods presented are carefully chosen to be generally applicable to most ecological questions.
As well as standard methods, this case study includes two methods for using machine learning methods with phylogenetic data that to the best of my knowledge have not been published before.
The included code is a fully reproducible analysis, with data being automatically downloaded, that can serve as a reference for how to implement the methods reviewed.



\closing{Yours sincerely,}

%----------------------------------------------------------------------------------------
%	OPTIONAL FOOTNOTE
%----------------------------------------------------------------------------------------

% Uncomment the 4 lines below to print a footnote with custom text
%\def\thefootnote{}
%\def\footnoterule{\hrule}
%\footnotetext{\hspace*{\fill}{\footnotesize\em Footnote text}}
%\def\thefootnote{\arabic{footnote}}

%----------------------------------------------------------------------------------------

\end{letter}

\end{document}
